\chapter*{Заключение}
\addcontentsline{toc}{chapter}{Заключение}

В ходе выполнения лабораторной работы были рассмотрены рекурсивный, матричный и рекурсивный с мемоизацией способы нахождения расстояний Левенштейна, а также матричная реализация алгоритма поиска Дамерау-Левенштейна. Была написана программа, реализующая эти алгоритмы. Для замера процессорного времени работы была выбрана функция $clock\_gettime()$. Также было проведено сравнение эффективности работы разных реализаций по времени. Наиболее эффективными оказались матричные реализации алгоритмов, при этом функции поиска расстояния Левенштейна работали быстрее, чем функции поиска расстояния Дамерау-Левенштейна.