% Содержимое отчета по курсу Анализ алгоритмов

\aaunnumberedsection{ВВЕДЕНИЕ}{sec:intro}

В данной лабораторной работе рассматривается организация параллельных вычислений по конвейерному принципу.

Цель работы -- получение навыка организации параллельных вычислений по конвейерному принципу. 

Задачи работы: 
\begin{itemize}
	\item анализ предметной области;
	\item разработка алгоритма обработки данных с выгруженных страниц сайта $eda.ru$~\cite{eda};
	\item создание ПО, реализующего разработанный алгоритм;
	\item исследование характеристик созданного ПО.
\end{itemize}

\aasection{Входные и выходные данные}{sec:input-output}

Входными данными является папка с $HTML$-файлами. Каждый файл содержит одну страницу рецепта, выгруженную с сайта $eda.ru$~\cite{eda}. Выходными данными является база данных, содержащая извлечённую из страниц информацию, а именно: $URL$ страницы рецепта, ингредиенты, шаги рецепта и $URL$ основного изображения рецепта. Каждая запись в базе данных дополнительно содержит уникальный идентификатор задачи, а также номер задачи из $Redmine$ (9134).

\aasection{Преобразование входных данных в выходные}{sec:algorithm}

Программа находит все файлы в папке. Из каждого файла считываются данные, после чего из них извлекаются $URL$ страницы рецепта, ингредиенты, шаги рецепта и $URL$ основного изображения рецепта. Полученные данные сохраняются в базе данных. Процесс завершается после обработки всех файлов в папке. 

\aasection{Примеры работы программы}{sec:demo}

Для реализации данной лабораторной работы был выбран язык $Java$, так как он содержит все необходимые средства для реализации алгоритмов. Нативные потоки создавались при помощи класса $Thread$~\cite{thread} через явный вызов конструктора. В качестве СУБД была выбрана $SQLite$, так как она соответствует требованиям из условия лабораторной работы~\cite{sqlite}.

На рисунке~\ref{img:console} представлен пример запуска программы из терминала. На рисунке~\ref{img:database} представлена таблица в базе данных с выгруженными данными. 

\FloatBarrier
\includeimage
{console} % Имя файла без расширения (файл должен быть расположен в директории inc/img/)
{f} % Обтекание (без обтекания)
{h} % Положение рисунка (см. figure из пакета float)
{1\textwidth} % Ширина рисунка
{Запуск программы} % Подпись рисунка
\FloatBarrier

\FloatBarrier
\includeimage
{database} % Имя файла без расширения (файл должен быть расположен в директории inc/img/)
{f} % Обтекание (без обтекания)
{h} % Положение рисунка (см. figure из пакета float)
{1\textwidth} % Ширина рисунка
{Полученная папка с рецептами} % Подпись рисунка
\FloatBarrier


\aasection{Тестирование}{sec:tests}

Выполнено тестирование программы по методологии чёрного ящика. В таблице~\ref{tab:tests} представлены функциональные тесты. Все тесты пройдены успешно.

\begin{longtable}{|p{.1\textwidth - 2\tabcolsep}|p{.3\textwidth - 2\tabcolsep}|p{.3\textwidth - 2\tabcolsep}|p{.3\textwidth - 2\tabcolsep}|}
	\caption{\label{tab:tests}Результаты выполнения функциональных тестов} \\
	\hline
	\makecell{№} & \makecell{Исходное\\количество\\файлов} & \makecell{Ожидаемое\\количество\\записей} & \makecell{Полученное\\количество\\записей} \\  
	\hline
	\makecell{1} & \makecell{10} & \makecell{10} & \makecell{10} \\
	\hline
	\makecell{2} & \makecell{15} & \makecell{15} & \makecell{15} \\
	\hline
	\makecell{3} & \makecell{25} & \makecell{25} & \makecell{25} \\
	\hline
	\makecell{4} & \makecell{100} & \makecell{100} & \makecell{100} \\
	\hline
\end{longtable}

\aasection{Описание исследования}{sec:study}

В ходе исследования требуется сформировать лог обработки задач, а также получить среднее время существования задачи, среднее время ожидания задачи в каждой из очередей, а также среднее время обработки задачи на каждой из стадий. Для формирования лога программа получала на вход папку с 10 выгруженными рецептами. Замеры времени проводились с помощью метода $System.nanoTime()$~\cite{nanotime}. Все замеры проводились на ноутбуке $Acer Swift 3x, процессор 11th Gen Intel(R) Core(TM) i7-1165G7$. 

Лог обработки задач представлен в таблице~\ref{tab:log}. Обозначения событий:
\begin{itemize}
	\item $created$ -- создание задачи;
	\item $start\_read$ -- начало чтения файла;
	\item $stop\_read$ -- окончание чтения файла;
	\item $start\_parse$ -- начало извлечения данных;
	\item $stop\_parse$ -- окончание извлечения данных;
	\item $start\_save$ -- начало сохранения данных;
	\item $stop\_save$ -- окончание сохранения данных;
	\item $destroyed$ -- уничтожение задачи;
\end{itemize}

\begin{longtable}{|p{.15\textwidth - 2\tabcolsep}|p{.35\textwidth - 2\tabcolsep}|p{.15\textwidth - 2\tabcolsep}|p{.35\textwidth - 2\tabcolsep}|}
	\caption{\label{tab:log}Лог обработки задач} \\
	\hline
	\makecell{№} & \makecell{Метка\\времени} & \makecell{Номер\\задачи} & \makecell{Событие} \\  
	\hline
	1 & 5914.566054132 & 1 & $created$\\
	\hline
	2 & 5914.567273217 & 2 & $created$\\
	\hline
	3 & 5914.567277993 & 1 & $start\_read$\\
	\hline
	4 & 5914.567498916 & 3 & $created$\\
	\hline
	5 & 5914.567775497 & 4 & $created$\\
	\hline
	6 & 5914.567929022 & 5 & $created$\\
	\hline
	7 & 5914.568164955 & 6 & $created$\\
	\hline
	8 & 5914.568311197 & 7 & $created$\\
	\hline
	9 & 5914.568466558 & 8 & $created$\\
	\hline
	10 & 5914.568661928 & 9 & $created$\\
	\hline
	11 & 5914.568818656 & 10 & $created$\\
	\hline
	12 & 5914.780051346 & 1 & $end\_read$\\
	\hline
	13 & 5914.780431769 & 2 & $start\_read$\\
	\hline
	14 & 5914.780494712 & 1 & $start\_parse$\\
	\hline
	15 & 5914.832295667 & 2 & $end\_read$\\
	\hline
	16 & 5914.832559352 & 3 & $start\_read$\\
	\hline
	17 & 5914.908876301 & 3 & $end\_read$\\
	\hline
	18 & 5914.909236768 & 4 & $start\_read$\\
	\hline
	19 & 5915.018138436 & 4 & $end\_read$\\
	\hline
	20 & 5915.018316692 & 5 & $start\_read$\\
	\hline
	21 & 5915.058474246 & 5 & $end\_read$\\
	\hline
	22 & 5915.058698517 & 6 & $start\_read$\\
	\hline
	23 & 5915.064677421 & 1 & $end\_parse$\\
	\hline
	24 & 5915.064829694 & 2 & $start\_parse$\\
	\hline
	25 & 5915.064893024 & 1 & $start\_save$\\
	\hline
	26 & 5915.092590470 & 6 & $end\_read$\\
	\hline
	27 & 5915.092770066 & 7 & $start\_read$\\
	\hline
	28 & 5915.126669438 & 7 & $end\_read$\\
	\hline
	29 & 5915.126906159 & 8 & $start\_read$\\
	\hline
	30 & 5915.175440624 & 8 & $end\_read$\\
	\hline
	31 & 5915.175696939 & 9 & $start\_read$\\
	\hline
	32 & 5915.229258635 & 9 & $end\_read$\\
	\hline
	33 & 5915.229924620 & 10 & $start\_read$\\
	\hline
	34 & 5915.260029949 & 1 & $end\_save$\\
	\hline
	35 & 5915.260302422 & 1 & $destroyed$\\
	\hline
	36 & 5915.263341521 & 10 & $end\_read$\\
	\hline
	37 & 5915.292567461 & 2 & $end\_parse$\\
	\hline
	38 & 5915.292742219 & 3 & $start\_parse$\\
	\hline
	39 & 5915.292771230 & 2 & $start\_save$\\
	\hline
	40 & 5915.304765031 & 2 & $end\_save$\\
	\hline
	41 & 5915.305110055 & 2 & $destroyed$\\
	\hline
	42 & 5915.535943401 & 3 & $end\_parse$\\
	\hline
	43 & 5915.536102866 & 4 & $start\_parse$\\
	\hline
	44 & 5915.536132931 & 3 & $start\_save$\\
	\hline
	45 & 5915.546228427 & 3 & $end\_save$\\
	\hline
	46 & 5915.546405107 & 3 & $destroyed$\\
	\hline
	47 & 5915.732401342 & 4 & $end\_parse$\\
	\hline
	48 & 5915.732542625 & 5 & $start\_parse$\\
	\hline
	49 & 5915.732543652 & 4 & $start\_save$\\
	\hline
	50 & 5915.742133994 & 4 & $end\_save$\\
	\hline
	51 & 5915.742264356 & 4 & $destroyed$\\
	\hline
	52 & 5915.866717891 & 5 & $end\_parse$\\
	\hline
	53 & 5915.866923255 & 5 & $start\_save$\\
	\hline
	54 & 5915.866924900 & 6 & $start\_parse$\\
	\hline
	55 & 5915.876488213 & 5 & $end\_save$\\
	\hline
	56 & 5915.876626804 & 5 & $destroyed$\\
	\hline
	57 & 5915.971145058 & 6 & $end\_parse$\\
	\hline
	58 & 5915.971294991 & 6 & $start\_save$\\
	\hline
	59 & 5915.971296282 & 7 & $start\_parse$\\
	\hline
	60 & 5915.976689610 & 6 & $end\_save$\\
	\hline
	61 & 5915.976842919 & 6 & $destroyed$\\
	\hline
	62 & 5916.075069522 & 7 & $end\_parse$\\
	\hline
	63 & 5916.075299060 & 8 & $start\_parse$\\
	\hline
	64 & 5916.075302356 & 7 & $start\_save$\\
	\hline
	65 & 5916.085196926 & 7 & $end\_save$\\
	\hline
	66 & 5916.085309083 & 7 & $destroyed$\\
	\hline
	67 & 5916.186589994 & 8 & $end\_parse$\\
	\hline
	68 & 5916.186721748 & 8 & $start\_save$\\
	\hline
	69 & 5916.186722932 & 9 & $start\_parse$\\
	\hline
	70 & 5916.197654520 & 8 & $end\_save$\\
	\hline
	71 & 5916.197762992 & 8 & $destroyed$\\
	\hline
	72 & 5916.307698707 & 9 & $end\_parse$\\
	\hline
	73 & 5916.307823192 & 10 & $start\_parse$\\
	\hline
	74 & 5916.307825498 & 9 & $start\_save$\\
	\hline
	75 & 5916.317253554 & 9 & $end\_save$\\
	\hline
	76 & 5916.317369350 & 9 & $destroyed$\\
	\hline
	77 & 5916.414392967 & 10 & $end\_parse$\\
	\hline
	78 & 5916.414509037 & 10 & $start\_save$\\
	\hline
	79 & 5916.420278978 & 10 & $end\_save$\\
	\hline
	80 & 5916.420386453 & 10 & $destroyed$\\
	\hline
\end{longtable}

В таблице~\ref{tab:mid} представлены результаты замеры среднего времени работы. Очередь 1 передаёт задачи из потока создания задач в поток чтения файлов, очередь 2 -- из потока чтения файлов в поток извлечения данных, очередь 3 -- из потока извлечения данных в поток сохранения данных, очередь 4 -- из потока сохранения данных в поток уничтожения задачи.

\begin{longtable}{|p{.05\textwidth - 2\tabcolsep}|p{.75\textwidth - 2\tabcolsep}|p{.2\textwidth - 2\tabcolsep}|}
	\caption{\label{tab:mid}Лог обработки задач} \\
	\hline
	\makecell{№} & \makecell{Характеристика} & \makecell{Значение} \\  
	\hline
	\makecell{1} & \makecell{Среднее время существования задачи} & \makecell{1.304942546} \\  
	\hline
	\makecell{2} & \makecell{Среднее время ожидания в очереди~1} & \makecell{0.411286480} \\  
	\hline
	\makecell{3} & \makecell{Среднее время ожидания в очереди~2} & \makecell{0.632964180} \\  
	\hline
	\makecell{4} & \makecell{Среднее время ожидания в очереди~3} & \makecell{0.000171396} \\  
	\hline
	\makecell{5} & \makecell{Среднее время ожидания в очереди~4} & \makecell{0.000166034} \\  
	\hline
	\makecell{6} & \makecell{Среднее время выполнения этапа чтения файлов} & \makecell{0.069331781} \\  
	\hline
	\makecell{7} & \makecell{Среднее время выполнения этапа извлечения данных} & \makecell{0.163242528} \\  
	\hline
	\makecell{8} & \makecell{Среднее время выполнения этапа сохранения данных} & \makecell{0.027780148} \\  
	\hline
\end{longtable}

Дольше всего выполнялся этап извлечения данных, из-за чего в очереди 2 среднее время ожидания дольше, чем в остальных, а в очередях 3 и 4 извлечение задачи происходило почти сразу же после постановки.

\aaunnumberedsection{ЗАКЛЮЧЕНИЕ}{sec:outro}

Цель работы достигнута. Решены все поставленные задачи: 
\begin{itemize}
	\item анализ предметной области;
	\item разработка алгоритма обработки данных с выгруженных страниц сайта $eda.ru$~\cite{eda};
	\item создание ПО, реализующего разработанный алгоритм;
	\item исследование характеристик созданного ПО.
\end{itemize}
