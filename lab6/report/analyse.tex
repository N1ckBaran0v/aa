\chapter{Аналитическая часть}

\section{Модель вычислений}

Модель вычислений описывает правила для задания оценки ресурсной эффективности алгоритмов.

Трудоёмкость следующих операций принимается за 1: $=$, $+$, $-$, $+=$, $-=$, $==$, $!=$, $<$, $<=$, $>$, $>=$, $[]$, $<<$, $>>$, $<<=$, $>>=$, $\&\&$, $||$, $\&$, $|$, $\textasciicircum$, $\&=$, $|=$. 

Трудоёмкость следующих операций принимается за 2: $*$, $/$, $\%$, $*=$, $/=$, $\%=$. 

Трудоёмкость функции возведения в степень $pow(a, b)$ принимается за 3.

Трудоёмкость цикла с известным количеством повторений рассчитывается следующим образом: пусть имеется цикл for (init; cond; inc) \{ body \}, где init --- блок инициализации переменных, cond --- блок проверки условия, inc --- блок изменения переменных, body --- тело цикла. Пусть цикл выполнился $N$ раз. Тогда трудоёмкость цикла $f_{for}$ рассчитывается по формуле~(\ref{eq:f_for}):

\begin{equation}
	\label{eq:f_for}
	f_{for} = f_{init} + f_{cond} + N \cdot (f_{cond} + f_{inc} + f_{body})
\end{equation}

где $f_{init}$ --- трудоёмкость блока инициализации, $f_{cond}$ --- трудоёмкость блока проверки условия, $f_{inc}$ --- трудоёмкость блока изменения переменных, $ f_{body}$ --- трудоёмкость тела цикла.

Трудоёмкость цикла с предусловием рассчитывается следующим образом: пусть имеется цикл while (cond) \{ body \}, где cond --- блок проверки условия,  body --- тело цикла. Пусть цикл выполнился $N$ раз. Тогда трудоёмкость цикла $f_{while}$ рассчитывается по формуле~(\ref{eq:f_while}).

\begin{equation}
	\label{eq:f_while}
	f_{while} = f_{cond} + N \cdot (f_{cond} + f_{body})
\end{equation}

где $f_{cond}$ --- трудоёмкость блока проверки условия, $ f_{body}$ --- трудоёмкость тела цикла.

Трудоёмкость условного оператора рассчитывается следующим образом: пусть имеется условный оператор if (cond) then \{ body1 \} else \{ body2 \}, где cond --- блок проверки условия, body1 --- тело при выполнении условия, body2 --- тело при невыполнении условия. Тогда трудоёмкость условного оператора $f_{if}$ рассчитывается по формуле~(\ref{eq:f_if}):

\begin{equation}
	\label{eq:f_if}
	f_{if} = f_{cond} + \begin{cases}
		f_{body1} & \text{; условие cond выполнилось}\\
		f_{body2} & \text{; иначе}\\
	\end{cases},
\end{equation}

где $f_{cond}$ --- трудоёмкость блока проверки условия, $ f_{body1}$ --- трудоёмкость тела при выполнении условия, $ f_{body2}$ --- трудоёмкость тела при невыполнении условия.

Операцией выделения памяти пренебрегаем.

\section{Постановка задачи}

Задача коммивояжёра состоит в поиске кратчайшего пути через все города на карте, в который каждый город входит ровно 1 раз~\cite{borozdov}. Согласно варианту, необходимо искать гамильтонов цикл, поэтому путь должен быть замкнутым~\cite{netcycle}. Также на рёбрах расположены затраты по времени перехода между вершинами в днях (измерение дней в вершинах не совпадает с измерением дней в муравьином алгоритме), при этом предел по времени --- 80 дней.

\section{Метод решения полным перебором}

В данном методе рассматриваются все возможные варианты пути~\cite{borozdov}. Преимуществом данного метода является гарантированное нахождение оптимального решения (если решение существует). Недостатком является временная сложность~$O(n!)$.

\section{Метод решения на основе муравьиного алгоритма}

Муравьиные алгоритмы основаны на имитации природных механизмов самоорганизации муравьёв~\cite{ants}. Метод заключается в поиске маршрута колонией муравьёв в течении $t$~дней.
Каждый день делится на 4 фазы --- утро, день, вечер и ночь. 

Утром муравьи покидают колонию и размещаются по 1 в каждом городе. В фазу <<день>> каждый из муравьёв пытается решить задачу. У каждого муравья есть 3 чувства, исходя из которых он выбирает маршрут:

\begin{itemize}
	\item память --- муравей не заходит в один и тот же город дважды;
	\item зрение --- находясь в городе, муравей может видеть все другие города, в которые можно перейти из данного города;
	\item обоняние --- муравей чувствует следы феромона на рёбрах, соединяющих города.
\end{itemize}

Вероятность перехода~$P$ муравья~$k$ из города~$i$ в город~$j$ в день~$t$ рассчитывается по формуле~(\ref{eq:pijkf}):

\begin{equation}
	\label{eq:pijkf}
	P_{ij,k}(t) = \begin{cases}
		\frac{[\tau_{ij}(t)]^\alpha\cdot[\eta_{ij}]^\beta}{\sum_{l \in J_{ik}} [\tau_{il}(t)]^\alpha\cdot[\eta_{il}]^\beta} &\text{; } j \in J_{ik}\\
		0 &\text{; иначе}\\
	\end{cases}
\end{equation}

\noindent где $\tau_{ij}(t)$ --- концентрация феромона на пути из города~$i$ в город~$j$ в день~$t$, $\eta_{ij}$ --- видимость города~$j$ из города~$i$, $J_{ik}$ --- множество городов, в которые можно попасть из города~$i$ и которые не были посещены муравьём~$k$, $\alpha \in [0, 1]$ --- коэффициент стадности, $\beta \in [0, 1]$ --- коэффициент жадности, причём $\alpha + \beta = 1$. Видимость~$\eta_{ij}$ рассчитывается по формуле~\ref{eq:etaij}:

\begin{equation}
	\label{eq:etaij}
	\eta_{ij} = \frac{1}{D_{ij}}
\end{equation}

\noindent где $D_{ij}$ --- расстояние из города~$i$ в город~$j$.

Вечером муравьи возвращаются в муравейник. Ночью для каждого пути из города~$i$ в город~$j$ обновляется концентрация феромона~$\tau_{ij}$ по формуле~(\ref{eq:tauijt}):

\begin{equation}
	\label{eq:tauijt}
	\tau_{ij}(t+1) = \tau_{ij}(t)\cdot(1-\rho)+\triangle\tau_{ij}(t)
\end{equation}

\noindent где $\triangle\tau_{ij}(t)$ --- изменение концентрации феромона на пути из города~$i$ в город~$j$ в день~$t$. $\triangle\tau_{ij}(t)$ рассчитывается по формуле~(\ref{eq:deltauijt}):

\begin{equation}
	\label{eq:deltauijt}
	\triangle\tau_{ij}(t) = \sum_{k=1}^{m}\triangle\tau_{ij,k}(t)
\end{equation}

\noindent где $\delta\tau_{ij,k}(t)$ --- концентрация феромона, которую оставил муравей~$k$ на пути из города~$i$ в город~$j$ в день~$t$, $m$ --- число муравьёв. $\triangle\tau_{ij,k}(t)$ рассчитывается по формуле~(\ref{eq:deltauijtk}):

\begin{equation}
	\label{eq:deltauijtk}
	\triangle\tau_{ij,k}(t) = \begin{cases}
		\frac{Q}{L_k(t)} &\text{;} (i, j) \in T_k(t)\\
		0 &\text{; иначе}\\
	\end{cases}
\end{equation}

\noindent где $Q$ --- количество феромона у муравья, значение которого выбирается одного порядка со значением оптимального маршрута, $L_k(t)$ --- длина маршрута, пройденного муравьём~$k$ в день~$t$, $T_k(t)$ --- маршрут, пройденный муравьём~$k$ в день~$t$. Кличество феромона~$Q$ можно рассчитать как отношение суммы всех дуг и удвоенного числа городов. Необходимо гарантировать необнуление феромона на ребре.

Преимуществом данного метода является временная сложность. Недостатком является отсутствие гарантии нахождения оптимального решения.

\section*{Выводы}

В данном разделе была описана постановка задачи коммивояжёра и методы её решения: метод полного перебора и метод, основанный на муравьином алгоритме.