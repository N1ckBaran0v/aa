\chapter{Технологическая часть}

\section{Средства реализации}

Для реализации данной лабораторной работы был выбран язык Java, так как он содержит необходимые средства для реализации алгоритмов. 

\section{Реализация алгоритмов}

В листингах \ref{lst:linear} и \ref{lst:binary} представлены реализации алгоритмов поиска в массиве. В листинге \ref{lst:sort} представлена реализация алгоритма пирамидальной сортировки.

\begin{lstlisting}[label=lst:linear,caption=Алгоритм линейного поиска в массиве]
public int search(int value) {
	var result = -1;
	for (var i = 0; result == -1 && i < array.length; ++i) {
		if (array[i] == value) {
			result = i;
		}
	}
	return returnsComprasionCount ? result == -1 ? array.length : result + 1 : result;
}
\end{lstlisting}

\begin{lstlisting}[label=lst:binary,caption=Алгоритм бинарного поиска в массиве]
public int binarySearch(int value) {
	var left = 0;
	var right = array.length - 1;
	var result = -1;
	var iterations = 0;
	while (result == -1 && left <= right) {
		var middle = left + ((right - left) >> 1);
		++iterations;
		if (array[middle] == value) {
			result = middle;
		} else if (value < array[middle]) {
			++iterations;
			right = middle - 1;
		} else {
			++iterations;
			left = middle + 1;
		}
	}
	return returnsComprasionCount ? iterations : result;
}
\end{lstlisting}

\begin{lstlisting}[label=lst:sort,caption=Алгоритм пирамидальной сортировки]
public static void sort(int[] array) {
	for (var i = array.length >> 1; i >= 0; --i) {
		heapify(array, i, array.length);
	}
	for (var i = array.length - 1; i > 0; --i) {
		swap(array, 0, i);
		heapify(array, 0, i);
	}
}

private static void heapify(int[] array, int index, int size) {
	while (index < (size >> 1)) {
		var left = (index << 1) + 1;
		var right = left + 1;
		if (right >= size) {
			if (left == size - 1 && array[index] < array[left]) {
				swap(array, index, left);
			}
			break;
		} else if (array[index] < array[left] && array[left] > array[right]) {
			swap(array, index, left);
			index = left;
		} else if (array[index] < array[right]) {
			swap(array, index, right);
			index = right;
		} else {
			break;
		}
	}
}

private static void swap(int[] array, int i, int j) {
	array[i] ^= array[j];
	array[j] ^= array[i];
	array[i] ^= array[j];
}
\end{lstlisting}

\section{Функциональные тесты}

В таблице \ref{tab:tests} представлены функциональные тесты. Тесты выполнялись для массива $[1, 2, 3, 4, 5, 6, 7, 8, 9, 10]$. Первое число в результате показывает результат выполнения линейного поиска, второе -- бинарного.

\begin{table}[h!]
	\small
	\caption{\label{tab:tests}Результаты выполнения функциональных тестов}
	\begin{center}
		\begin{tabular}{|c|c|c|c|c|c|}
			\hline
			№  & \makecell{Число} & \makecell{Ожидаемый\\вывод} & \makecell{Полученный\\вывод} & \makecell{Описание теста}\\  
			\hline
			1  & 1 & 0 0 & 0 0 & \makecell{Первое число}\\  
			\hline
			2  & 10 & 9 9 & 9 9 & \makecell{Последнее число}\\  
			\hline
			3  & 7 & 6 6 & 6 6 & \makecell{Иное число}\\  
			\hline
			4  & 11 & -1 -1 & -1 -1 & \makecell{Не найдено}\\  
			\hline
		\end{tabular}
	\end{center}
\end{table}

\section{Вывод}

В данном разделе был выбран язык программирования для написания программы, были рализованы все ранее описанные алгоритмы и описаны функциональные тесты.