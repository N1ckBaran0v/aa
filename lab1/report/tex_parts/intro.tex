\chapter*{ВВЕДЕНИЕ}
\addcontentsline{toc}{chapter}{ВВЕДЕНИЕ}

В данной лабораторной работе рассматривается расстояние Левенштейна между двумя строками -- это минимальное количество операций вставки одного символа, удаления одного символа и замены одного символа на другой, необходимых для превращения строки в другую~\cite{levenshtein}. 

Расстояние Левенштейна применяется для решения следующих задач:
\begin{itemize}
	\item для исправления ошибок в слове поискового запроса;
	\item в формах заполнения информации на сайтах;
	\item для распознавания рукописных символов;
	\item в базах данных.
\end{itemize}

Расстояние Дамерау-Левенштейна отличается наличием ещё одной операции транспозиции (перестановки двух соседних символов).

Цель: исследование алгоритмов поиска расстояний Левенштейна и Даме\-рау-Левенштейна.

Задачи:
\begin{enumerate}[label={\arabic*)}]
	\item описать различные алгоритмы поиска расстояний Левенштейна и Даме\-рау-Левенштейна;
	\item написать программу, реализующую несколько версий алгоритма поиска расстояния Левенштейна и одну версию алгоритма поиска расстояния Дамерау-Левенштейна;
	\item выбрать инструменты для замера процессорного времени выполнения реализаций алгоритмов;
	\item провести анализ затрат реализаций алгоритмов по времени с использованием микропроцессора $STM32$.
\end{enumerate}