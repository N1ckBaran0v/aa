\chapter*{Введение}
\addcontentsline{toc}{chapter}{Введение}

В данной лабораторной работе будет рассмотрено hасстояние Левенштейна между двумя строками - это минимальное количество операций вставки одного символа, удаления одного символа и замены одного символа на другой, необходимых для превращения строки в другую\cite{levenshtein}. 

Расстояние Левенштейна применяется для решения следующих задач:
\begin{itemize}
	\item для исправления ошибок в слове поискового запроса;
	\item в формах заполнения информации на сайтах;
	\item для распознавания рукописных символов;
	\item в базах данных.
\end{itemize}

Расстояние Дамерау-Левенштейна отличается наличием операции транспозиции (перестановки двух соседних символов).

Цель: описание и исследование алгоритмов поиска расстояний Левенштейна и Дамерау-Левенштейна.

Задачи:
\begin{enumerate}[label={\arabic*)}]
	\item Описать алгоритмы поиска расстояний Левенштейна и Дамерау-Левенштейна;
	\item Написать программу, реализующую несколько версий алгоритма поиска расстояния Левенштейна и одну версию алгоритма поиска расстояния Дамерау-Левенштейна;
	\item Выбрать инструменты для замера процессорного времени выполнения реализаций алгоритмов;
	\item Провести анализ затрат реализаций алгоритмов по времени.
\end{enumerate}