\chapter{Аналитическая часть}

\section{Расстояние Левенштейна}

Рассмотрим 3 возможных реализации алгоритма - матричную, рекурсивную и рекурсивную с мемоизацией.

\subsection{Рекурсивный алгоритм поиска расстояния Левенштейна}

Пусть $S_1$ и $S_2$ - 2 строки (длиной n и m соответственно) над некоторым алфавитом \cite{levenshtein2}. Тогда расстояние Левенштейна $d(S_1, S_2)$ можно подсчитать по формуле \ref{eq:D}. 

\begin{equation}
	\label{eq:D}
	D[i, j] = \begin{cases}
		0 &\text{; i = 0, j = 0}\\
		i &\text{; j = 0, i > 0}\\
		j &\text{; i = 0, j > 0}\\
		\min \lbrace \\
		\qquad D[i, j-1]+ 1 &\text{; i > 0, j > 0}\\
		\qquad D[i-1, j] + 1 \\
		\qquad D[i-1, j-1] + m(S_1[i], S_2[j]) \\
		\rbrace \\
	\end{cases},
\end{equation}

Функция \ref{eq:m} показывает, была ли осуществлена замена символа.

\begin{equation}
	\label{eq:m}
	m(a, b) = \begin{cases}
		0 &\text{если a = b,}\\
		1 &\text{иначе}
	\end{cases}.
\end{equation}

На основе формулы \ref{eq:D} получается рекурсивный алгоритм.

\subsection{Матричный алгоритм поиска расстояния Левенштейна}

При реализации функции с использованием формулы \ref{eq:D} может возникнуть ситуация, при которой будет возникать повторное вычисление значения для одних и тех же входных параметров. Для решения этой проблемы можно создать матрицу для сохранения промежуточных значений, после чего построчно заполнить её.

\subsection{Рекурсивный алгоритм поиска расстояния Левенштейна с мемоизацией}

Данный алгоритм получается путем объединении двух предыдущих алгоритмов. Он также использует рекурсию, но вычисленные значения сохраняются в таблице, что позволяет избежать повторного вычисления одних и тех же значений.

\section{Расстояние Дамерау-Левенштейна}

Расстояние Дамерау-Левенштейна отличается от расстояния Левенштейна тем, что для него также определена операция перестановки двух соседних символов. Для его вычисления к результату, полученному с помощью формулы\ref{eq:D} необходимо применить формулу\ref{eq:d}\cite[levenshtein]:

\begin{equation}
	\label{eq:d}
	D[i, j]= \begin{cases}
		\min \lbrace & \text{; Если выполняются условия} \\
		\qquad D[i, j] & \text{; i > 1, j > 1} \\
		\qquad D[i-2, j-2] + m(S_1[i], S_2[j]) & \text{; $S_1[i]$ = $S_2[j-1]$} \\
		\rbrace & \text{; $S_1[i-1]$ = $S_2[j]$} \\
		D[i, j] & \text{; Если не выполняются условия}\\
	\end{cases}
\end{equation}

Здесь также могут быть реализованы 3 алгоритма - рекурсивный, матричный и рекурсивный с мемоизацией.

\section{Вывод}

В данном разделе были рассмотрены 3 метода для вычисления расстояний Левенштейна и Дамерау-Левенштейна. 




