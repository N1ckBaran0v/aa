% пакеты
\documentclass[a4paper, 14pt]{extreport}

\usepackage{cmap}
\usepackage[utf8]{inputenc}
\usepackage[T2A]{fontenc}
\usepackage[english,russian]{babel}
\usepackage{amssymb,amsfonts,amsmath,mathtext,enumerate,float}
\usepackage{pgfplots}
\usepackage{graphicx}
\usepackage{tocloft}
\usepackage{listings}
\usepackage{caption}
\usepackage{tempora}
\usepackage{titlesec}
\usepackage{setspace}
\usepackage{geometry}
\usepackage{indentfirst}
\usepackage{pdfpages}
\usepackage{enumerate,letltxmacro}
\usepackage{threeparttable}
\usepackage[unicode,pdftex]{hyperref}
\hypersetup{hidelinks}
\usepackage{flafter}
\usepackage{enumitem}
\usepackage{multirow}

\usepackage[figure,table]{totalcount}
\usepackage{lastpage}

\setlist{nosep}

\usepackage{titlesec}
\titleformat{\section}{\normalsize\bfseries}{\thesection}{1em}{}
\titleformat{name=\chapter,numberless}[block]{\hspace{\parindent}}{}{0pt}{\large\bfseries\centering}
\titlespacing*{\chapter}{12.5mm}{-30pt}{8pt}
\titlespacing*{\section}{\parindent}{*4}{*4}
\titlespacing*{\subsection}{\parindent}{*4}{*4}

\usepackage{titlesec}
\titleformat{\chapter}{\large\bfseries}{\thechapter}{20pt}{\large\bfseries}
\titleformat{\section}{\large\bfseries}{\thesection}{20pt}{\large\bfseries}

\makeatletter
\renewcommand{\@biblabel}[1]{#1.}
\makeatother
%
%\titleformat{\chapter}[hang]{\LARGE\bfseries}{\hspace{1.25cm}\thechapter}{1ex}{\LARGE\bfseries}
%\titleformat{\section}[hang]{\Large\bfseries}{\hspace{1.25cm}\thesection}{1ex}{\Large\bfseries}
%\titleformat{name=\section,numberless}[hang]{\Large\bfseries}{\hspace{1.25cm}}{0pt}{\Large\bfseries}
%\titleformat{\subsection}[hang]{\large\bfseries}{\hspace{1.25cm}\thesubsection}{1ex}{\large\bfseries}
%\titlespacing{\chapter}{0pt}{-\baselineskip}{\baselineskip}
%\titlespacing*{\section}{0pt}{\baselineskip}{\baselineskip}
%\titlespacing*{\subsection}{0pt}{\baselineskip}{\baselineskip}

\geometry{left=30mm}
\geometry{right=10mm}
\geometry{top=20mm}
\geometry{bottom=20mm}

\onehalfspacing

\renewcommand{\theenumi}{\arabic{enumi}}
\renewcommand{\labelenumi}{\arabic{enumi}\text{)}}
\renewcommand{\theenumii}{.\arabic{enumii}}
\renewcommand{\labelenumii}{\asbuk{enumii}\text{)}}
\renewcommand{\theenumiii}{.\arabic{enumiii}}
\renewcommand{\labelenumiii}{\arabic{enumi}.\arabic{enumii}.\arabic{enumiii}.}

\renewcommand{\cftchapleader}{\cftdotfill{\cftdotsep}}

\addto\captionsrussian{\renewcommand{\figurename}{Рисунок}}
\DeclareCaptionLabelSeparator{dash}{~---~}
\captionsetup{labelsep=dash}

\captionsetup[figure]{justification=centering,labelsep=dash}
\captionsetup[table]{labelsep=dash,justification=raggedright,singlelinecheck=off}

\graphicspath{{images/}}%путь к рисункам

\newcommand{\floor}[1]{\lfloor #1 \rfloor}

\usepackage{color}

\lstset{ %
	language=java,                 % выбор языка для подсветки (здесь это Java)
	basicstyle=\small\ttfamily, % размер и начертание шрифта для подсветки кода
	numbers=none,               % где поставить нумерацию строк (слева\справа)
	numberstyle=\small,           % размер шрифта для номеров строк
	stepnumber=1,                   % размер шага между двумя номерами строк
	%	numbersep=5pt,                % как далеко отстоят номера строк от подсвечиваемого кода
	showspaces=false,            % показывать или нет пробелы специальными отступами
	showstringspaces=false,      % показывать или нет пробелы в строках
	showtabs=false,             % показывать или нет табуляцию в строках
	frame=single,              % рисовать рамку вокруг кода
	tabsize=4,                 % размер табуляции по умолчанию равен 2 пробелам
	captionpos=t,              % позиция заголовка вверху [t] или внизу [b] 
	breaklines=true,           % автоматически переносить строки (да\нет)
	breakatwhitespace=false, % переносить строки только если есть пробел
	escapeinside={\#*}{*)},   % если нужно добавить комментарии в коде
	abovecaptionskip=-5pt                   % Показать название подгружаемого файла
}

\pgfplotsset{width=0.85\linewidth, height=0.5\columnwidth}

\linespread{1.3}

\parindent=1.25cm

%\LetLtxMacro\itemold\item
%\renewcommand{\item}{\itemindent0.75cm\itemold}

\def\labelitemi{---}
\setlist[itemize]{leftmargin=1.25cm, itemindent=0.65cm}
\setlist[enumerate]{leftmargin=1.25cm, itemindent=0.55cm}

\newcommand{\specialcell}[2][c]{%
	\begin{tabular}[#1]{@{}c@{}}#2\end{tabular}}

\frenchspacing

\usepackage{makecell}

% скобки после пунктов
\setenumerate[1]{label={ \arabic*)}}
\usepackage{enumitem}

% математика в листингах
\lstset{
mathescape=true
}

% \FloatBarrier
\usepackage{placeins}

% $ в листингах
\newcommand{\dlr}{\mbox{\textdollar}}

% Тире в itemize
\renewcommand\labelitemi{---}

% \foreach
\usepackage{pgffor}

% Длинные таблицы
\usepackage{longtable}

% Картинки
\newcommand{\img}[3] {
\begin{figure}[t!]
    \center{\includegraphics[height=#1]{inc/img/#2}}
    \caption{#3}
    \label{img:#2}
\end{figure}
}
\newcommand{\imgw}[3] {
\begin{figure}[h!]
    \center{\includegraphics[width=#1]{inc/img/#2}}
    \caption{#3}
    \label{img:#2}
\end{figure}
}
\newcommand{\boximg}[3] {
\begin{figure}[h]
    \center{\fbox{\includegraphics[height=#1]{inc/img/#2}}}
    \caption{#3}
    \label{img:#2}
\end{figure}
}

% АА в сокращенном виде

\newcommand{\aasection}[2] {
    \vspace{20mm}
    {\let\clearpage\relax \chapter{#1}\label{#2}}
}

\newcommand{\aaunnumberedsection}[2] {
    \vspace{20mm}
    \addcontentsline{toc}{chapter}{#1}
    {\let\clearpage\relax \chapter*{#1}\label{#2}}
}

\bibliographystyle{gost-numeric.bbx}
\usepackage[parentracker=true,
backend=biber,
hyperref=false,
bibencoding=utf8,
style=numeric-comp,
language=auto,
autolang=other,
citestyle=gost-numeric,
defernumbers=true,
bibstyle=gost-numeric,
sorting=ntvy,
]{biblatex}
\addbibresource{lib.bib}
