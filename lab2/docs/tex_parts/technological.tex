\chapter{Технологическая часть}

\section{Средства реализации}

Для реализации данной лабораторной работы был выбран язык C++, так как он содержит необходимые средства для реализации алгоритмов. 

\section{Реализация алгоритмов}

В листингах \ref{lst:standart}, \ref{lst:standartOpt}, \ref{lst:vinograd}, \ref{lst:vinogradOpt}  представлены реализации алгоритмов перемножения матриц.

\begin{lstlisting}[label=lst:standart,caption=Стандартный алгоритм перемножения матриц]
Matrix StandartAlgorithm::execute() {
	auto c = initMatrix(m, q);
	for (auto i = 0; i < m; ++i) {
		for (auto j = 0; j < q; ++j) {
			c[i][j] = 0;
			for (auto k = 0; k < n; ++k) {
				c[i][j] = c[i][j] + a[i][k] * b[k][j];
			}
		}
	}
	return c;
}
\end{lstlisting}

\begin{lstlisting}[label=lst:standartOpt,caption=Стандартный алгоритм с оптимизациями]
Matrix StandartOptimisedAlgorithm::execute() {
	auto c = initMatrix(m, q);
	for (auto i = 0; i < m; ++i) {
		for (auto j = 0; j < q; ++j) {
			c[i][j] = a[i][0] * b[0][j];
			for (auto k = 1; k < n; ++k) {
				c[i][j] += a[i][k] * b[k][j];
			}
		}
	}
	return c;
}
\end{lstlisting}

\begin{lstlisting}[label=lst:vinograd,caption=Алгоритм Винограда]
Matrix VinogradAlgorithm::execute() {
	auto c = initMatrix(m, q);
	for (auto i = 0; i < m; ++i) {
		mulR[i] = 0;
		for (auto k = 0; k < n / 2; ++k) {
			mulR[i] = mulR[i] + a[i][2 * k] * a[i][2 * k + 1];
		}
	}
	for (auto j = 0; j < q; ++j) {
		mulC[j] = 0;
		for (auto k = 0; k < n / 2; ++k) {
			mulC[j] = mulC[j] + b[2 * k][j] * b[2 * k + 1][j];
		}
	}
	for (auto i = 0; i < m; ++i) {
		for (auto j = 0; j < q; ++j) {
			c[i][j] = -mulR[i] - mulC[j];
			for (auto k = 0; k < n / 2; ++k) {
				c[i][j] = c[i][j] + (a[i][2 * k] + b[2 * k + 1][j]) * (a[i][2 * k + 1] + b[2 * k][j]);
			}
			if (n % 2 == 1) {
				c[i][j] = c[i][j] + a[i][n - 1] * b[n - 1][j];
			}
		}
	}
	return c;
}
\end{lstlisting}

\begin{lstlisting}[label=lst:vinogradOpt,caption=Алгоритм Винограда с оптимизациями]
Matrix VinogradOptimisedAlgorithm::execute() {
	auto c = initMatrix(m, q);
	for (auto i = 0; i < m; ++i) {
		if (n > 1) {
			mulR[i] = a[i][0] * a[i][1];
		} else {
			mulR[i] = 0;
		}
		for (auto k = 1; k < n / 2; ++k) {
			mulR[i] += a[i][k << 1] * a[i][(k << 1) + 1];
		}
	}
	for (auto j = 0; j < q; ++j) {
		if (n > 1) {
			mulC[j] = b[0][j] * b[1][j];
		} else {
			mulC[j] = 0;
		}
		for (auto k = 1; k < n / 2; ++k) {
			mulC[j] += b[k << 1][j] * b[(k << 1) + 1][j];
		}
	}
	for (auto i = 0; i < m; ++i) {
		for (auto j = 0; j < q; ++j) {
			if (n > 1) {
				c[i][j] = (a[i][0] + b[1][j]) * (a[i][1] + b[0][j]) - mulR[i] - mulC[j];
			} else {
				c[i][j] = -mulR[i] - mulC[j];
			}
			for (auto k = 1; k < n / 2; ++k) {
				c[i][j] += (a[i][k << 1] + b[(k << 1) + 1][j]) * (a[i][(k << 1) + 1] + b[k << 1][j]);
			}
			if (n % 2 == 1) {
				c[i][j] += a[i][n - 1] * b[n - 1][j];
			}
		}
	}
	return c;
}
\end{lstlisting}

\section{Функциональные тесты}

В таблице \ref{tab} представлены функциональные тесты. Для матриц a и b сначала вводятся их размеры. В результате каждый из алгоритмов успешно прошел все тесты.

\begin{table}[h!]
	\small
	\caption{\label{tab}Результаты выполнения функциональных тестов}
	\begin{center}
		\begin{tabular}{|c|c|c|c|c|c|}
			\hline
			№  & \makecell{Матрица a} & \makecell{Матрица b} & \makecell{Результат\\(матрица c\\или ошибка)} & \makecell{Описание теста}\\  
			\hline
			1  & -1 &  & \makecell{Ошибка:\\Некорректный\\размер} & \makecell{Ошибка в\\размере}\\ 
			\hline
			2  & \makecell{2 2\\2 2\\2 2} & \makecell{3 2} & \makecell{Ошибка:\\Операция\\недопустима} & \makecell{Разные\\размеры}\\ 
			\hline
			3  & \makecell{1 1\\3} & \makecell{1 1\\3} & \makecell{9} & \makecell{Минимальный\\размер}\\ 
			\hline
			4  & \makecell{2 2\\1 2\\3 4} & \makecell{2 2\\5 6\\7 8} & \makecell{19 22\\43 50} & \makecell{Квадратные\\матрицы}\\ 
			\hline
			5  & \makecell{3 2\\1 2\\3 4\\5 6} & \makecell{2 4\\1 2 3 4\\5 6 7 8} & \makecell{11 14 17 20\\23 30 37 44\\35 46 57 68} & \makecell{Разные\\матрицы}\\ 
			\hline
		\end{tabular}
	\end{center}
\end{table}

\section{Оценка ресурсной эффективности}

Так как по модели вычислений операции выделения памяти не учитываются, то стоимость строки создания матрицы c принимается за 0. Также не учитывается стоимость возврата значения.

Для стандартного алгоритма перемножения матриц (листинг \ref{lst:standart}) трудоёмкость равна $1 + 1 + m * (1 + 1 + 1 + 1 + q * (1 + 1 + 3 + 1 + 1 + n * (1 + 1 + 12))) = 2 + m * (4 + q * (7 + 14n) = 14mnq + 7mq + 4m + 2$.

Для стандартного алгоритма с оптимизациями (листинг \ref{lst:standartOpt}) трудоёмкость равна $1 + 1 + m * (1 + 1 + 1 + 1 + q * (1 + 1 + 10 + 1 + 1 + (n - 1) * (1 + 1 + 9))) = 2 + m * (4 + q * (3 + 11n) = 11mnq + 3mq + 4m + 2$.

Для алгоритма Винограда (листинг \ref{lst:vinograd}) трудоёмкость равна $1 + 1 + m * (1 + 1 + 1 + 1 + 3 + \frac{n}{2} * (1 + 3 + 15)) + 1 + 1 + q * (1 + 1 + 1 + 1 + 3 + \frac{n}{2} * (1 + 3 + 15)) + 1 + 1 + m * (1 + 1 + 1 + 1 + q * (1 + 1 + 6 + 1 + 3 + \frac{n}{2} * (1 + 3 + 28) + 3 + 
	\begin{cases}
		0 & \text{, n четное, лучший случай (л. с.)}\\
		14 & \text{, n нечетное, худший случай (х. с.)} \\
	\end{cases}
	)) = 2 + 7m + \frac{19n}{2} + 2 + 7q + \frac{19n}{2} + 2 + m * (4 + q * (12 + 16n + 
	\begin{cases}
		0 & \text{, л. с.}\\
		14 & \text{, х. с.} \\
	\end{cases}
	)) = 16mnq + 
	\begin{cases}
		12 & \text{, л. с.}\\
		26 & \text{, х. с.} \\
	\end{cases} mq + \frac{19nq}{2} + \frac{19mn}{2} + 11m + 7q + 6$.

В следующем расчете предполагается, что $n = 1$ (минимальный общий размер у матриц). Это позволяет избавиться от рассмотрения лишних случаев в трёх условных операторах. Для алгоритма Винограда с оптимизациями (листинг \ref{lst:vinogradOpt}) трудоёмкость равна $1 + 1 + m * (1 + 1 + 7 + 1 + 3 + (\frac{n}{2} - 1) * (1 + 3 + 11)) + 1 + 1 + q * (1 + 1 + 7 + 1 + 3 + (\frac{n}{2} - 1) * (1 + 3 + 11)) + 1 + 1 + m * (1 + 1 + 1 + 1 + q * (1 + 1 + 20 + 1 + 3 + (\frac{n}{2} - 1) * (1 + 3 + 21) + 3 + 
\begin{cases}
	0 & \text{, n четное, лучший случай (л. с.)}\\
	11 & \text{, n нечетное, худший случай (х. с.)} \\
\end{cases}
)) = 2 - 2m + \frac{15n}{2} + 2 - 2q + \frac{15n}{2} + 2 + m * (4 + q * (-13 + \frac{25n}{2} + 
\begin{cases}
	0 & \text{, л. с.}\\
	11 & \text{, х. с.} \\
\end{cases}
)) = \frac{25mnq}{2} - 
\begin{cases}
	13 & \text{, л. с.}\\
	2 & \text{, х. с.} \\
\end{cases} mq + \frac{15nq}{2} + \frac{15mn}{2} + 2m - 2q + 6$.

Если принять m, n и q равными, то при больших размерах матрицы наименее трудоёмким должен быть оптимизированный стандартный алгоритм, и только затем оптимизированный алгоритм Винограда. Это связано с тем, что в последнем не были решены главные проблемы, такие как деление на 2 на каждой итерации, а также умножение индексов на 2 было заменено на побитовый сдвиг, а не убрано.

\section{Вывод}

В данном разделе был выбран язык программирования для написания программы, были рализованы все ранее описанные алгоритмы, описаны тесты, а также произведена оценка ресурсной эффективности алгоримтов.