\chapter{Конструкторская часть}

\section{Алгоритм, использующий полный перебор}

На рисунке~\ref{img:full} представлена схема алгоритма, использующего полный перебор. На вход он получает количество городов~n и матрицу смежности~matrix размером~n~на~n. В случае отсутствия ребра между городами в матрицу смежности в соответствующую ячейку вместо значения времени перехода заносится -1.

\FloatBarrier
\includeimage
{full} % Имя файла без расширения (файл должен быть расположен в директории inc/img/)
{f} % Обтекание (без обтекания)
{h!} % Положение рисунка (см. figure из пакета float)
{0.85\textwidth} % Ширина рисунка
{Схема алгоритма, использующего полный перебор} % Подпись рисунка
\FloatBarrier

В худшем случае дороги есть между всеми городами. Трудоёмкость блока начальной инициализации данных равна 5. Трудоёмкость блока проверки нового цикла в случае обновления составляет 21, в случае сохранения старой длины --- 20, при этом количество возможных рассматриваемых циклов равно $\frac{(n-1)!}{2}$, при этом каждый цикл проверяется 2 раза в зависимости от направления. Трудоёмкость блока выставления курсора равна 6, при этом курсор выставляется $(n-1)!$ раз. Трудоёмкость блока возврата равна 12 при возврате на нулевой индекс и 16 в противном случае, при этом возврат на нулевой индекс происходит $n$ раз, на остальные --- $(n-2)!$ раз. Трудоёмкость блока входа в новый город составляет 25, вход в новый город происходит $(n-1)!$ раз. Трудоёмкость игнорирования города составляет 14, игнорирование города происходит $(n-2) \cdot (n-1)!$ раз. Тогда трудоёмкость поиска полным перебором в худшем случае~$f_{worst\_complete}$ равна $5 + 20.5 \cdot (n-1)! + 6 \cdot (n-1)! + 12 \cdot n + 16 \cdot (n-2)! + 25 \cdot (n-1)! + 14 \cdot (n-2) \cdot (n-1)! + 1 = 14 \cdot n! - 1.5 \cdot (n-1)! + 16 \cdot (n-2)! + 12 \cdot n + 1$.

\section{Муравьиный алгоритм}

На рисунках~\ref{img:ants0}-\ref{img:ants2} представлена схема муравьиного алгоритма. На вход он получает количество городов~N, матрицу смежности~matrix размером~N~на~N, коэффициент стадности~alpha, коэффициент испарения~rho и количество дней~days.

\FloatBarrier
\includeimage
{ants0} % Имя файла без расширения (файл должен быть расположен в директории inc/img/)
{f} % Обтекание (без обтекания)
{h!} % Положение рисунка (см. figure из пакета float)
{0.45\textwidth} % Ширина рисунка
{Схема муравьиного алгоритма (начало)} % Подпись рисунка
\FloatBarrier

\FloatBarrier
\includeimage
{ants1} % Имя файла без расширения (файл должен быть расположен в директории inc/img/)
{f} % Обтекание (без обтекания)
{h!} % Положение рисунка (см. figure из пакета float)
{0.45\textwidth} % Ширина рисунка
{Схема муравьиного алгоритма (продолжение)} % Подпись рисунка
\FloatBarrier

\FloatBarrier
\includeimage
{ants2} % Имя файла без расширения (файл должен быть расположен в директории inc/img/)
{f} % Обтекание (без обтекания)
{h!} % Положение рисунка (см. figure из пакета float)
{0.8\textwidth} % Ширина рисунка
{Схема муравьиного алгоритма (окончание)} % Подпись рисунка
\FloatBarrier

\clearpage

В худшем случае дороги есть между всеми городами. Трудоёмкость муравьиного алгоритма~$f_{worst\_ants}$ рассчитывается по формуле~(\ref{eq:f_a0}):

\begin{equation}
	\label{eq:f_a0}
	f_{worst\_ants} = f_{prep} + 2 + days \cdot (2 + f_{morning} + f_{day} + f_{evening} + f_{night}) + 1
\end{equation}

\noindent где $f_{prep}$ --- трудоёмкость этапа подготовки данных, $f_{morning}$ --- трудоёмкость фазы утра, $f_{day}$ --- трудоёмкость фазы дня, $f_{evening}$ --- трудоёмкость фазы вечера, $f_{night}$ --- трудоёмкость фазы ночи. 

Трудоёмкость этапа подготовки данных~$f_{prep}$ равна $4 + 2 + n \cdot (2 + 2 + n \cdot (2 + 11)) + 4 = 13 \cdot n^2 + 4 \cdot n + 10$. Трудоёмкость фазы утра~$f_{morning}$ равна $2 + n \cdot (2 + 4 + 2 + n \cdot (2 + 15)) = 17 \cdot n ^ 2 + 8 \cdot n + 2$. Трудоёмкость фазы дня~$f_{day}$ в худшем случае равна $2 + n \cdot (2 + 2 + n \cdot 4 + 2 + (n - 1) \cdot 31 + \frac{n(n - 1)}{2} \cdot 11 + 20) = 5.5 \cdot n^3 + 29.5 \cdot n^2 - 7 \cdot n + 2$. Трудоёмкость фазы вечера~$f_{evening}$ в худшем случае равна $9 \cdot n + 2$. Трудоёмкость фазы ночи~$f_{night}$ в худшем случае равна $2 + n \cdot (2 + 4 + 2 + n \cdot (2 + 16) + 16) + 2 + 4 \cdot (n - 1) + \frac{n(n - 1)}{2} \cdot 26 - n \cdot 3 = 18 \cdot n^2 + 24 \cdot n + 2 + 13 \cdot n^2 - 7 \cdot n - 2 = 31 \cdot n^2 + 17 \cdot n$.

Трудоёмкость муравьиного алгоритма~$f_{worst\_ants}$ в худшем случае равна $13 \cdot n^2 + 4 \cdot n + 10 + 2 + days \cdot (2 + 17 \cdot n ^ 2 + 8 \cdot n + 2 + 5.5 \cdot n^3 + 29.5 \cdot n^2 - 7 \cdot n + 2 + 9 \cdot n + 2 + 31 \cdot n^2 + 17 \cdot n) = 13 \cdot n^2 + 4 \cdot n + 12 + days \cdot (5.5 \cdot n^3 + 77.5 \cdot n^2 + 2 \cdot n + 8)$.

\section*{Выводы}

В данном разделе были построены схемы алгоритмов, а также была выполнена оценка трудоёмкости алгоритмов.

